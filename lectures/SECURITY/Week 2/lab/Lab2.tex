\documentclass{article}

\usepackage[utf8]{inputenc}
\usepackage{hyperref}
\usepackage{listings}

\title{Lab Sheet 2}
\date{Security}

\begin{document}
\maketitle
In our second lab, we will work on classical encryption techniques in Kali Linux. Boot your Kali Linux installation and use a terminal to invoke shell commands for encryption and decryption. The starting point of each task is an open terminal. When finished post a screenshot of your answers to slack.
\section{Task 1: Caesar Cipher}
For this task, you will need to install the Python libray pycipher by typing \texttt{pip install pycipher} into your terminal.
\subsection{Encryption with a known key}
Type \texttt{python} to invoke the Python interpreter and use \texttt{pycipher} to encrypt the message \textit{m = 'hello world'} with key \textit{k = 3}.
HINT: use \texttt{help(pyciper)} to find the right function.
\end{document}
